\documentclass[reqno,psamsfonts, 10pt]{amsart}
\usepackage{graphicx, amssymb, amsmath, amsfonts, hyperref, amsthm, amsbsy}
\usepackage{mathrsfs}
\usepackage[active]{srcltx}

\newtheorem{thm}{Theorem}
\newtheorem{cor}[thm]{Corollary}
\newtheorem{lem}[thm]{Lemma}
\newtheorem{prop}[thm]{Proposition}
\newtheorem{definition}[thm]{Definition}

\theoremstyle{remark}
\newtheorem{rem}[thm]{Remark}

\newcommand{\cov}{\mbox{$\mathrm{cov}$}}
\newcommand{\sgn}{\mbox{$\mathrm{sgn}$}}
\newcommand{\Id}{\mbox{$\mathrm{Id}$}}
\newcommand{\sinc}{\mbox{$\mathrm{sinc}$}}
\newcommand{\comment}[1]{}
\newtheorem{conj}{Conjecture}

%\numberwithin{equation}{section}


\bigskip
\bigskip
\bigskip
\bigskip
\bigskip
\bigskip
\bigskip
\bigskip
\bigskip

\begin{document}

\section*{Stochastic Calculus, Spring 2025, March 17, \\
Lecture 1\\
Construction of Brownian motion}

\footnotetext[1]{this version \today}

Reading for this lecture
(for references see the end of the lecture):
\begin{itemize}
\item \cite{1} pp. 72-108
\end{itemize}
\subsection*{Scaled Random Walks}

Our main object of study in this course will be
\textit{stochastic processes}. Loosely speaking, stochastic
process is a collection of random variables indexed by
``time" variable $t$: $X(t).$ Variable $t$ could take
values in a discreet set, for instance in a set of
positive integers or it could take values in a continuous
set, for instance in $[0,\infty).$ An example of stochastic
process is a stock price that changes in time.

Our main objective for today is to build a very important
stochastic process called \textbf{Brownian Motion}. Brownian
Motion is a continuous-time stochastic process usually indexed by
$[0,\infty).$ One dimensional Brownian Motion satisfies the
following properties:
\begin{description}
\item[\bf Independence of increments.] If $t_0 < t_1 < \dots <
t_n$ then random variables $B(t_0), B(t_1) - B(t_0), \dots, B(t_n)
- B(t_{n-1})$ are independent.

\item[Normal distribution of increments.] If $s, t \ge 0$
then
\begin{equation}
\mathbb{P}\left[ B(t+s) - B(s) \in A) \right] = \int \limits_A
\frac{1}{\sqrt{2 \pi t}} e^{-x^2/2t} dx.
\end{equation}

\item[Continuity of trajectories.] With probability one,
$B_0 = 0$ and $t \to B_t$ is continuous.
\end{description}




It is not obvious from the above definition that Brownian
Motion exists. We show that it does exist by essentially
constructing Brownian Motion.

We start with a simpler stochastic process that is called
\textbf{Symmetric Random Walk}~(SRW).  To construct a SRW we
repeatedly toss a fair coin: $\mathbb{P}(H) = \mathbb{P}(T) =
1/2.$ We define the successive outcomes of the tosses by $w_1,
w_2, w_3 \dots \in \{H,T\}$ and define $w = w_1 w_2 w_3 \dots,$ in
other words, $w$ is the infinite sequence of tosses and $w_n$ is
the outcome of the $n^{\mbox{th}}$ toss. Next, let $X_j$ be a
random variable defined as
\begin{equation}
X_j =
\begin{cases}
1  \quad & \mbox{ if } w_j = \mbox{H,} \nonumber \\
-1 \quad & \mbox{ if } w_j = \mbox{T.} \nonumber
\end{cases}
\end{equation}
Define discrete-time stochastic process $M(n)$ in the
following way: $$M(0) = 0, M(n) = \sum \limits_{j=1}^{n}
X_j.$$ The process $M(n)$ is called a symmetric random
walk. With each step it either steps up one unit or down
one unit, and each of the two possibilities is equally
likely.

\bigskip



%\textbf{Exercise 1.} For a fixed positive integers $k, n$ what is
%the probability $\mathbb{P}(M(n) = k)?$

\subsection*{Increments of the SRW} A random walk has
independent increments. This means that if we choose nonnegative
integers $0 = n_0 < n_1 < \dots < n_k,$ then random variables
$$
M(n_1) = M(n_1) - M(n_0), M(n_2) - M(n_1), \dots, M(n_k) -
M(n_{k-1})
$$
are independent. Each of these random variables
$$
M(n_{i+1}) - M(n_i) = \sum \limits_{j = n_i + 1}^{n_{i+1}}
X_j
$$
is called an increment of the random walk. It is the change
in the position of the random walk between times $n_i$ and
$n_{i+1}.$ Increments over non-overlapping time intervals
are independent because they depend on different coin
tosses. Moreover,
$$
\mathbb{E}(M(n_{i+1}) - M(n_i)) = \sum \limits_{j = n_i
+1}^{n_{i+1}} \mathbb{E} X_j = 0
$$
and
$$
\mbox{Var} (M(n _{i+1}) - M(n_i)) = \sum \limits_{j = n_i
+1}^{n_{i+1}} \mbox{Var} X_j = n_{i+1} - n_i \ \mbox{since}
\ \mbox{Var} X_j = 1.
$$



\subsection*{Martingale property for the SRW}


In order to define martingale we need the notion of
$\sigma$-algebra. There is a very important, nontechnical
reason to include $\sigma$-algebras in the study of
stochastic processes, and that is to keep track of
information. The temporal feature of a stochastic process
suggests a flow of time, in which, at every moment $t \ge
0,$ we can talk about a past, present and future and can
ask how much an observer of the process knows about it at
present, as compared to how much he knew at some point in
the past or will know at some point in the future.
\begin{definition} A filtration on $(\Omega, \mathcal{F})$
is a family $\{\mathcal{F}_t\}_{t \ge 0}$ of
$\sigma$-algebras $\mathcal{F}_t \subset \mathcal{F}$ such
that
$$
0 \le s \le t \Longrightarrow \mathcal{F}_s \subset
\mathcal{F}_t, \mbox{i.e., } \{\mathcal{F}_t\}_{t \ge 0}
\mbox{ is increasing}.
$$
\end{definition}
Thus, informally, one can think about the $\sigma$-algebra
$\mathcal{F}_t$ in the above definition as a ``knowledge"
available at time $t.$ The fact that $\{\mathcal{F}_t\}_{t
\ge 0}$ is increasing just reflects the fact that as time
goes on the amount of available information increases.


\begin{definition}
A stochastic process $\{Z_t\}_{t \ge 0}$ is called a
martingale with respect to a filtration
$\{\mathcal{F}_t\}_{t \ge 0} $ if
\begin{description}
\item[a)] $Z_t$ is $\mathcal{F}_t$-measurable for all
$t$

\item[b)] $\mathbb{E}|Z_t| < \infty$ for all $t$

\item[c)] $\mathbb{E}(Z_t | \mathcal{F}_s) = Z_s$ for all
$s \le t.$
\end{description}
\end{definition}

The most important property in the above definition is property
$c).$ Simply put, it states that ``the best" guess for $X_t$ given
information available at moment $s$ is~$X_s,$ and thus process
$X_t$ has neither upward nor downward drift.

Now let's check that the SRW is a martingale. First, we need to
specify a filtration $\{ \mathcal{F}_t \}_{t \ge 0}.$ Let
$\mathcal{F}_n$ be the a $\sigma$-algebra(read information)
generated by the set of random variables $X_1, X_2, \dotsm X_n.$
\begin{description}
\item[a)] $M(n) = \sum \limits_{j=1}^n X_j$ is
$\mathcal{F}_n$-measurable since it depends only on $X_j, j
\le n,$ i.e. on the information available at time $n.$

\item[b)] From the fact that $|M(t)| \le t$
$$
\left(\mathbb{E}|M(t)|\right)^2 \le t < \infty.
$$

\item[c)] using the fact that $\mathbb{E}(M(s) |
\mathcal{F}_s) = M(s)$ and the fact that  random variable
$M(t) - M(s)$ is independent of $\sigma$-algebra
$\mathcal{F}_s$ we can write:
\begin{align}
\mathbb{E}(M(t) | \mathcal{F}_s) & = \mathbb{E}(M(t) - M(s)
+ M(s)| \mathcal{F}_s) = \mathbb{E}(M(t) - M(s)|
\mathcal{F}_s)
+ \mathbb{E}(M(s) | \mathcal{F}_s) \nonumber \\
& = \mathbb{E}(M(t) - M(s)) + M(s) = 0 + M(s) = M(s).
\end{align}

\end{description}



\subsection*{Quadratic variation of the SRW}

Quadratic variation of a discrete stochastic process up to time
$t$ is defined as
\begin{equation}
\langle M,M\rangle_t = \sum \limits_{j=1}^t(M(j) -
M(j-1))^2.
\end{equation}
The quadratic variation up to time $t$ along a path is computed by
taking all the one-step increments $M(j) - M(j-1)$ along that
path, squaring these increments and then summing them. Clearly,
for the symmetric random walk increments can take values $\pm 1$
and thus
\begin{equation}
\langle M,M\rangle_t = t.
\end{equation}
Note that this is computed path by path and the fact that
quadratic variation is the same along any path is a special
feature of the symmetric random walk we consider. The
consequence of the fact that quadratic variation is
computed path by path is that for a general stochastic
process quadratic variation is a random quantity, depending
on the trajectory.



\subsection*{Scaled SRW}
To approximate a Brownian motion we speed up time and scale
down the step size of a SRW. More precisely, we fix a
positive integer $n$ and define the scaled SRW at rational
points $\frac{k}{n}$ as
\begin{equation}\label{scaledSRW}
B^{(n)}\left(\frac{k}{n}\right) = \frac{1}{\sqrt{n}} M(k).
\end{equation}
At all other points we define $B^{(n)}(t)$ be linear
interpolation between its values at the nearest points of
the form $\frac{k}{n}.$


The following properties of the scaled SRW could be easily
proved using the corresponding properties of the SRW and we
leave their proof as an exercise:

\begin{description}
\item[a)] independence of increments, i.e., for all
rational numbers $0 = t_0 < t_1 < \dots < t_n$ of the form
$\frac{k}{n}$ random variables
\begin{equation}
B^{(n)}(t_1) - B^{(n)}(t_0), B^{(n)}(t_2) - B^{(n)}(t_1),
\dots, B^{(n)}(t_n) - B^{(n)}(t_{n-1})
\end{equation}
are independent(because they depend on different coin
tosses).

\item[b)] $\mathbb{E}(B^{(n)}(t) - B^{(n)}(s)) = 0,
\mbox{Var} (B^{(n)}(t) - B^{(n)}(s)) = t - s.$

\item[c)] $\mathbb{E}(B^{(n)}(t)| \mathcal{F}_s) =
B^{(n)}(s).$

\item[d)] quadratic variation of the scaled SRW:
\begin{equation*}
\langle B^{(n)} B^{(n)}\rangle_t = t.
\end{equation*}

\end{description}


\subsection*{Limiting distribution of the scaled SRW}

Now the idea is to let $n \to \infty$ in \eqref{scaledSRW}
and in the limit obtain Brownian motion. In the previous
section we studied a single path(trajectory) of the scaled
SRM. In other words, we have fixed a sequence of coin
tosses $w = w_1 w_2 \dots$ and drawn the path of the
resulting process as time $t$ varies. Another way to think
about the scaled symmetric random walk is to fix time $t$
and consider the set of all possible paths evaluated at
that time $t,$ i.e., consider the distribution of random
variable $B^{(n)}(t).$ By definition $B^{(n)}(t) =
\frac{1}{\sqrt{n}} M_{nt}$ or if $t = \frac{k}{n}$ then
$B^{(n)}(t) = \frac{1}{\sqrt{n}} M_k.$ Since $M_k = \sum
\limits_{j=1}^k X_j$ has a binomial distribution with
parameters $k$ and $1/2$ one can easily calculate the
distribution of $B^{(n)}(t).$ In particular, one can draw
the histogram of $B^{(n)}(t).$

We know that random variable $B^{(n)}(t)$ has mean $0$ and
variance $t.$ If we draw on top of the histogram of
$B^{(n)}(t)$ the graph of normal density with mean zero and
variance $t$ we will see that the distribution of
$B^{(n)}(t)$ is nearly normal. Actually, the Central Limit
Theorem asserts that $B^{(n)}(t)$ converges in distribution
to the normal random variable $N(0,t).$ But let us not use
this theorem and prove the convergence of $B^{(n)}(t)$ from
scratch. The whole point of this exercise is learning a
very powerful mathematical tool known as ``characteristic
functions".

\begin{thm}
For any fixed $t \ge 0,$ the distribution of the scaled
random walk $B^{(n)}(t)$ evaluated at time $t$ converges to
the normal distribution with mean zero and variance~$t.$
\end{thm}


\begin{proof}
To prove this theorem we use a tool from probability theory
- characteristic functions.
\begin{definition}
Let $X$ be a random variable with distribution
$\mathbb{P}.$ Characteristic function of $X$ is defined as
\begin{equation}
\varphi_X(u) = \mathbb{E}e^{i u X} = \int e^{iux} dP(x)
\end{equation}
for every $u \in \mathbb{R}.$
\end{definition}

\textbf{Example.}

1) $X = Be(p).$ Then
\begin{equation}
\varphi_X(u) = p(e^{iu} - 1) + 1.
\end{equation}

2) $X = Bi(n,p).$ Then $\mathbb{P}(X=k) = C_n^k
p^k(1-p)^{n-k}$ and
\begin{equation}
\varphi_X(u) = (p(e^{iu} - 1) + 1)^n.
\end{equation}

3) $X = \mbox{Poisson}(\lambda),$ then $\mathbb{P}(X = k) =
e^{-\lambda} \frac{\lambda^k}{k!}$ and
\begin{equation}
\varphi_X(u) = \sum \limits_{k=0}^\infty e^{-\lambda}
\frac{\lambda^k}{k!} e^{i u k} = e^{-\lambda} \sum
\limits_{k=0}^{\infty} \frac{(\lambda e^{iu})^k }{k!} =
e^{\lambda(e^{it} - 1)}.
\end{equation}

4) $X$ is a Gaussian random variable $N(m,\sigma^2).$ Then
\begin{equation}
\varphi_X(u) = e^{iu m - u^2 \sigma^2/2}.
\end{equation}


The importance of the characteristic functions consists
could be summarized in the following two theorems:

1. If two random variables have the same characteristic
function then they have the same distribution.(of course,
if two random variables have the same distributions then
they have the same characteristic functions).

2. \textbf{Paul L\'{e}vy Theorem.} Let $X_n$ be a sequence
of random variables on $\mathbb{R}.$ Then

a) If $X_n$ converges to $X$ is distribution then
$\varphi_{X_n}(u) \to \varphi_X(u)$ for every $u$ and in
fact this convergence is uniform.

b) If $\varphi_{X_n}(u) \to \varphi_X(u)$ then $X_n$
converges to $X$ is distribution.

c) If $\varphi_{X_n}(u)$ converges to some function
$\varphi(u)$ and $\varphi(u)$ is continuous at $0$ then
there exists a unique random variable $X$ such that
$\varphi$ is the characteristic function of $X$ and $X_n$
converges to $X$ in distribution.


We use part (b) of the theorem to prove that $B^{(n)}(t)$
converges to normal distribution with expectation zero and
variance $t.$

\begin{align}
\varphi_{B^{(n)}}(u) & = \mathbb{E} e^{i u B^{(n)}(t)} =
\mathbb{E} e^{i u \frac{1}{\sqrt{n}} M_{nt} } = \mathbb{E}
e^{i u \frac{1}{\sqrt{n}} \sum \limits_{j=1}^{nt} X_j } =
\left(\mathbb{E} e^{i u \frac{1}{\sqrt{n}} X_j
}\right)^{nt} \nonumber \\
& = \left(\frac{1}{2} e^{i u /\sqrt{n}} + \frac{1}{2}
e^{-iu/\sqrt{n}}\right)^{nt}.
\end{align}
We need to show that as $n \to \infty$
$$
\varphi_n(u) = \left(\frac{1}{2} e^{i u /\sqrt{n}} +
\frac{1}{2} e^{-iu/\sqrt{n}}\right)^{nt}
$$
converges to the characteristic function of the normal
random variable with mean $0$ and variance $t,$ i.e., to
$e^{-u^2 t/2}.$ Expanding in Taylor series
\begin{align}
\frac{1}{2} e^{i u /\sqrt{n}} + \frac{1}{2}
e^{-iu/\sqrt{n}} = 1 - \frac{u^2}{2n} +
O(\frac{1}{n^{3/2}})
\end{align}
and thus
$$
\varphi_n(u) = \left(1 - \frac{u^2}{2n} +
O(\frac{1}{n^{3/2}})\right)^{nt} \to e^{-\frac{u^2}{2n}
\cdot nt} = e^{-u^2 t/2}.
$$
Thus we have proved that $B^{(n)}(t)$ converges in
distribution to a normal random variable with expectation
zero and variance $t.$

\begin{rem} Truly speaking, we have just demonstrated that
for a fixed moment $t$ distribution of $B^{(n)}(t)$
converges to $N(0,t)$ but we did not prove the convergence
of the whole stochastic process to Brownian motion. To do
this one has to prove the convergence of the finite
dimensional distributions and use Prokhorov's theorem.
\end{rem}
\end{proof}




\begin{thebibliography}{1}

\bibitem[1]{1} Steven Shreve, \textit{Stochastic Calculus for Finance II:
Continuous-Time Models}

\bibitem[2]{2} Richard Durrett, \textit{Stochastic Calculus: A Practical Introduction.}

\bibitem[3]{3} Ioannis Karatzas, Steven E. Shreve, \textit{Brownian Motion and Stochastic Calculus.}

\bibitem[4]{4} Bernt Oksendal, Stochastic Differential Equations
\end{thebibliography}
\end{document}
