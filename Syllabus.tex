\documentclass[reqno,psamsfonts, 12pt]{amsart}
\usepackage{graphicx, amssymb, amsmath, amsfonts, hyperref, amsthm, amsbsy}
\usepackage{mathrsfs}
\usepackage[active]{srcltx}

\newtheorem{thm}{Theorem}[section]
\newtheorem{cor}[thm]{Corollary}
\newtheorem{lem}[thm]{Lemma}
\newtheorem{prop}[thm]{Proposition}

\theoremstyle{remark}
\newtheorem{rem}[thm]{Remark}

\def\urltilda{\kern -.15em\lower .7ex\hbox{\~{}}\kern .04em}

\newcommand{\cov}{\mbox{$\mathrm{cov}$}}
\newcommand{\sgn}{\mbox{$\mathrm{sgn}$}}
\newcommand{\Id}{\mbox{$\mathrm{Id}$}}
\newcommand{\sinc}{\mbox{$\mathrm{sinc}$}}
\newcommand{\comment}[1]{}
\newtheorem{conj}{Conjecture}

\numberwithin{equation}{section}

\bigskip
\bigskip
\bigskip
\bigskip
\bigskip
\bigskip
\bigskip
\bigskip
\bigskip

\begin{document}

\section*{MATH-GA.2903-001, Stochastic Calculus: Spring 2025\\
New York University, Courant Institute}

\bigskip
\bigskip

\noindent \textbf{Instructor:} Oleg Shorokhov, \emph{oleg.shorokhov@nyu.edu}\\

\noindent \textbf{Teaching Assistant:} Xinqiao Tong, \emph{xinqiao.tong@nyu.edu}

\bigskip
\bigskip

\noindent \textbf{Course Content:} Will be available through NYU Brightspace.

\bigskip
\bigskip

\noindent \textbf{Class Time:} Mondays 5:10-7:00pm @ CIWW 201

\bigskip
\bigskip

\noindent \textbf{Prerequisites:} Basic Probability or Equivalent

\bigskip
\bigskip

\noindent \textbf{Textbooks:} There is no required textbook for the course. 
The lecture material will be self-contained and enough to complete the homework assignments and exams. 
However, I recommend that students have access to the following books:

\begin{itemize}
\item \emph{Stochastic Calculus for Finance II: Continuous-Time Models} (Springer Finance) (v.2) by Steven E. Shreve.

\medskip

\item \emph{Brownian Motion and Stochastic Calculus (Graduate Texts in Mathematics)} by Ioannis Karatzas and Steven E. Shreve

\medskip

\item \emph{Stochastic Calculus: A Practical Introduction (Probability and Stochastics Series)} by Richard Durrett.

\medskip

\item \emph{Stochastic Differential Equations: An Introduction with Applications (Universitext)} by Bernt Oksendal.
\end{itemize}

\bigskip
\bigskip

\noindent \textbf{Syllabus:}

\begin{description}

\item[Week 1] Bernoulli trials, weak convergence of rescaled random walk to Brownian Motion.

\item[Week 2] Brownian motion and its simplest properties.

\item[Week 3] Construction of Ito integral.

\item[Week 4] Ito formula.

\item[Week 5] Solution of simple SDE's (Black-Scholes SDE, Ornstein-Uhlenbeck process).

\item[Week 6] Cameron-Martin Theorem, application to Monte Carlo and Brownian motion with drift.

\item[Week 7] Connection of stochastic calculus and partial differential equations. Feynman-Kac formula.

\item[Week 8] Final Exam.

\end{description}

\bigskip
\bigskip

\noindent \textbf{Assignments:} Every week there will be a new homework assignment and you will have exactly one week to complete it. Late assignments will NOT be accepted.

\bigskip
\bigskip

\noindent \textbf{Exams:} There will be a in-class final exams. Any student who is unable to take an exam must have a very good reason for doing so, e.g., a medical emergency.

\bigskip
\bigskip

\noindent \textbf{Grading:} The grading scheme is: Assignments 30\%, Final 70\%.

Students are permitted to exchange ideas on the homework but are not permitted to turn in joint or essentially duplicate solutions.
Both midterm and final exams will be in-class.

\bigskip
\bigskip

\noindent \textbf{Academic Integrity:} Students must make a serious commitment to academic integrity. 
The consequences of cheating are serious for the cheater and for NYU as a whole. 
Students are required to immediately report cheating incidents they observe to the instructors.

A student caught cheating on an assignment may have their grade for the class reduced by one letter at the first offense and their grade reduced to an F, or dismissal from NYU, for repeated offenses. 
During exams and quizzes, students may not communicate in any way, nor use any materials or technology not explicitly permitted. 
No cellphone or other electronic devices may be used during the exam; they should be stored away. 
Students may not look at each other’s test and/or screen during the exam. Cheating on an exam or quiz will immediately result in an F for the class.

Students are not permitted to share course materials outside of class without written permission of the instructors. 
Students are not allowed to record (photography, audio and/or video) any lectures or sessions without written permission of the instructors.
All disciplinary actions will be taken in accordance with the policies of the NYU Graduate School of Arts and Science: GSAS Statement on Academic Integrity (nyu.edu)

\end{document}
